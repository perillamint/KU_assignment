%!TEX TS-program = xelatex

\documentclass {article}

\usepackage{xetexko}
\usepackage[a4paper]{geometry}
\usepackage[usenames,dvipsnames]{xcolor}
\usepackage{mathtools}
\usepackage{amsmath}
\usepackage{fontspec}
\usepackage{hyperref}
\usepackage{graphicx}
\usepackage{listings}
\usepackage{makeidx}
\usepackage{indentfirst}
\usepackage{tikz}
\usetikzlibrary{arrows,automata}


%\setmainfont {NanumMyeongjo}
\setmainfont {UnBatang}
\setmonofont[Scale=0.8]{DejaVu Sans Mono}

\lstdefinestyle{diff}{
  belowcaptionskip=1\baselineskip,
  breaklines=true,
  frame=L,
  xleftmargin=\parindent,
  showstringspaces=false,
  % Diffstart
  morecomment=[f][\color{gray}]{@@},
  % Diffincl
  morecomment=[f][\color{Green}]{+},
  % Diffrem
  morecomment=[f][\color{Red}]{-},
  basicstyle=\footnotesize\ttfamily,
}

\lstdefinestyle{customtxt}{
  belowcaptionskip=1\baselineskip,
  breaklines=true,
  frame=L,
  xleftmargin=\parindent,
  showstringspaces=false,
  basicstyle=\footnotesize\ttfamily,
}

\lstdefinestyle{customc}{
  belowcaptionskip=1\baselineskip,
  breaklines=true,
  frame=L,
  xleftmargin=\parindent,
  language=C,
  showstringspaces=false,
  basicstyle=\footnotesize\ttfamily,
  keywordstyle=\bfseries\color{green!40!black},
  commentstyle=\itshape\color{purple!40!black},
  identifierstyle=\color{blue},
  stringstyle=\color{orange},
}

\lstdefinestyle{customrs}{
  belowcaptionskip=1\baselineskip,
  breaklines=true,
  frame=L,
  xleftmargin=\parindent,
  showstringspaces=false,
  morekeywords={fn,let,mut,pub,use,impl,struct,unsafe,if,for},
  morecomment=[l]{//},
  morecomment=[n]{/*}{*/},
  basicstyle=\footnotesize\ttfamily,
  keywordstyle=\bfseries\color{green!40!black},
  commentstyle=\itshape\color{purple!40!black},
  identifierstyle=\color{blue},
  stringstyle=\color{orange},
}


\begin {document}

\title {동양인과 서양인의 구분 조건}
\input {../../reportauthor.tex}
\maketitle

\section {전제}
이 구분에서는, 국적이 아닌, 인종을 기준으로 나누는 것을 전제로 한다. 또한, 주어진 문제에서는,
세상의 인종을 동양인과, 서양인 두 카테고리로 나누는 것으로 제한했으며, 이를 이용하면, 동양인에
속하지 않는다면, 서양인이라는 논리를 세울 수 있게 된다.
\section {동양인일 수 있는 조건들}
\subsection {동양인이란}
여기서는, 편의상 동양인을 Mongoloid 로 한정하여 생각하기로 한다.
\subsection {동양인(몽골로이드) 의 외형적 특징들}
동양인의 외형적 특징은 개략적으로 다음과 같다.
\begin {itemize}
\item 이마가 넓다
\item 코뼈가 작다
\item 앞니가 삽 모양이다
\item 광대뼈가 튀어나와 있다
\end{itemize}
\subsection {외형적 특징 이외의 다른 동양인과 서양인을 구분짓는 특징은?}
이러한 외형적 특징 외에도, 동양인의 문화는 흔히 CJK 문화권이라고 일컬어지는, 중국이 문화 중심이
되었었던 시절의 문화가 남아 있다. 이를 이용해서, 만약 외모 외에, 구별하려는 사람이 쓴 글이나,
그림과 같은 매체가 주어지면, 이를 이용해서 구분할 수도 있을 것이다.

특히, 한국어의 경우, 영어에 비해, 수식어의 빈도가 낮으므로, 이를 이용한다면, 만약 구분하려는
대상이, 다른 언어로 글을 쓴다고 해도 이를 구분해 낼 수 있을 것이다. 
\section {구분을 실제 구현하려면?}
외형적 특징을 이용해서 구분하는 것은, 위에 주어진 특징을 이용하기보단, 인공 신경망을 이용해서,
수천 개 가량의 동양인과 서양인으로 이루어진 이미지를 공급해서, 신경망을 훈련시킨다. 이는,
사람이 직접 이미지의 프로세싱의 조건을 코드로 작성하는 것보다, 컴퓨터에게 이미지의 특징을
뽑아내는 것을 맡김으로서, 인간이 놓치거나 오판할 수 있는 부분을 쉽게 고려할 수 있게 된다.

문화적 요소 중 글을 이용하는 방법의 경우는, 언어를 분석해서, 단어들을 분류한 후, 각 언어권의
특성에 맞게 통계를 낸 뒤, 글의 단어 빈도를 이용해서, 이를 구분해낼 수 있을 것이다.

\end {document}
