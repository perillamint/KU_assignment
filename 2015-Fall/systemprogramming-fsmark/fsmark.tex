%!TEX TS-program = xelatex

\documentclass {article}

\usepackage{xetexko}
\usepackage[a4paper]{geometry}
\usepackage[usenames,dvipsnames]{xcolor}
\usepackage{mathtools}
\usepackage{amsmath}
\usepackage{fontspec}
\usepackage{hyperref}
\usepackage{graphicx}
\usepackage{listings}
\usepackage{makeidx}
\usepackage{indentfirst}


%\setmainfont {NanumMyeongjo}
\setmainfont {UnBatang}
\setmonofont[Scale=0.8]{DejaVu Sans Mono}

\input {../../lststyles.tex}

\begin {document}

\title {Ext3 과 NILFS2 의 Block I/O 패턴 비교}
\input {../../reportauthor.tex}
\maketitle

\section{Ext3 과 NILFS2 의 비교}
Ext3 파일 시스템은 저널링 파일 시스템이다. Ext3 은 기본적으로 저널에 메타정보와 파일 내용을 기록 후, 이를 디스크 영역에 적어넣는다. 반면,
NILFS2 파일 시스템은 Log-structed 파일 시스템이다. 이는 로그를 연속된 블럭으로 저장하는 방식으로 데이터를 적어넣는다.

이 두 파일 시스템의 차이가 만들어내는 Block write 패턴은, 아래의 벤치마크 결과에서 Block I/O 패턴의 차이로 나타나게 된다.
\section{Block I/O 결과}
\subsection{사용된 환경}
\setlength{\parindent}{0cm}
툴체인
\begin{lstlisting}[style=customtxt]
  arm-eabi-4.6 from Android Open-source Project
\end{lstlisting}
원본 커널 소스 코드 리버전
\begin{lstlisting}[style=customtxt]
Tizen TRATS2 kernel code commit c349d3966a91b3036d9e4a16a7849ab23d1e6435
(branch tizen_2.3, retrieved from Tizen gerrit code review, identical as TA version)
\end{lstlisting}
iozone 옵션
\begin{lstlisting}[style=customtxt]
  iozone -a -i 0 -n 32m -g 32m -r 16k
\end{lstlisting}
FS image size: 각각 256MiB
\newpage
\subsection{결과}
\subsubsection{Ext3}
\begin{figure} [h!]
    \caption{ext3}
    \begin{center}
      \resizebox{0.5\textwidth}{!}{
        \input{iozone_ext3.tex}
      }
    \end{center}
\end{figure}
Ext3 의 경우, 저널링 파일 시스템의 특성을 Block I/O 결과에서 나타내고 있다. 블럭에 쓰기를 할 때, 2블럭씩 저널에 쓴 뒤, 이를 실제 디스크 영역에 작성하게 된다.

따라서, 파일 쓰기 그래프가 순차적으로 쓰여지지 않고, 끊어지게 된다.
\subsubsection{NILFS2}
\begin{figure} [h!]
  \caption{nilfs2}
  \begin{center}
    \resizebox{0.5\textwidth}{!}{
      \input{iozone_nilfs2.tex}
    }
  \end{center}
\end{figure}
NILFS2 의 경우, Log-structed 파일 시스템의 특성을 나타내고 있다. NILFS write 의 경우, segctord 데몬이, 쓰기 요청을 모아, 약 248블럭씩 로그에 지속적으로 작성하는 것을 볼 수 있다.

이는 Ext3 의 Block I/O 패턴과 다르게, 순차적으로 파일이 쓰여지는 패턴을 보이게 된다.

\newpage
\section{부록}
\subsection{소스 코드}
\lstinputlisting [style=customc]{srcs/bio-benchmark.c}
\subsection{Patchset}
\setlength{\parindent}{0cm}
0001-Fix-compile-error-due-to-missing-sys-ioccom.h.-https.patch
\lstinputlisting [style=diff]{patchset/0001-Fix-compile-error-due-to-missing-sys-ioccom.h.-https.patch}
0002-Samsung-s-fallback-does-not-work.-Since-we-does-not-.patch
\lstinputlisting [style=diff]{patchset/0002-Samsung-s-fallback-does-not-work.-Since-we-does-not-.patch}
0003-sudo-without-password-might-be-bad-idea.patch
\lstinputlisting [style=diff]{patchset/0003-sudo-without-password-might-be-bad-idea.patch}
0004-path-fix.patch
\lstinputlisting [style=diff]{patchset/0004-path-fix.patch}
0005-Block-I-O-benchmark-code.patch
\lstinputlisting [style=diff]{patchset/0005-Block-I-O-benchmark-code.patch}
\end {document}
