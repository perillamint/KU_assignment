%!TEX TS-program = xelatex

\documentclass {article}

\usepackage{xetexko}
\usepackage[a4paper]{geometry}
\usepackage[usenames,dvipsnames]{xcolor}
\usepackage{mathtools}
\usepackage{amsmath}
\usepackage{fontspec}
\usepackage{hyperref}
\usepackage{graphicx}
\usepackage{listings}
\usepackage{makeidx}
\usepackage{indentfirst}


%\setmainfont {NanumMyeongjo}
\setmainfont {UnBatang}
\setmonofont[Scale=0.8]{DejaVu Sans Mono}

\input {../../lststyles.tex}

\begin {document}

\title {Socket sharing}
\input {../../reportauthor.tex}
\maketitle

\section{소켓의 생성}
소켓은 socket(2) 시스템 콜을 호출함으로서 생성되게 된다. 이 시스템 콜의 결과로, 소켓의 파일 디스크립터가 반환되게 되고, 이는 해당 프로세스에 열린 파일의 형태로 소켓을 제공하게 된다.

이 결과로, 소켓이 프로세스에 종속되게 됨을 의미하고, 프로세스에게는 파일의 형태로 추상화된 네트워크 소켓이 주어지게 된다.

\section{다중 접속 서버의 생성}
다중 접속 서버의 경우, fork(2) 콜을 이용해서 프로세스의 복제를 만드는 기법이 전통적으로 사용되어 왔다. fork(2) 콜을 사용함으로서 해당 프로세스의 열린 파일을 포함한 기타 상태를 공유하는 복제를 생성하게 된다.

이를 이용하여, 부모 프로세스는, 소켓에 bind() 한 뒤, listen() 해서, 새 접속을 기다리게 되고, 새 접속이 들어오게 되면, fork() 를 하여, 자식 프로세스가 복제된 파일 디스크립터를 가지고, 입출력을 하여 통신을 하게 된다. 따라서, 부모 프로세스는 계속 새 연결을 기다리면서, 새 자식 프로세스를 fork() 해 낼 수 있게 된다.

\section{부록 - fork() 를 사용한 HTTP 서버 예제 코드}
inc0gnito 발표에서 사용했던 서버 코드 - \url{https://github.com/perillamint/K-HomeWRT-hack/}
\lstinputlisting [style=customc]{granelver.c}

\end {document}
