%!TEX TS-program = xelatex

\documentclass {article}

\usepackage{xetexko}
\usepackage[a4paper]{geometry}
\usepackage[usenames,dvipsnames]{xcolor}
\usepackage{mathtools}
\usepackage{amsmath}
\usepackage{fontspec}
\usepackage{hyperref}
\usepackage{graphicx}
\usepackage{listings}
\usepackage{makeidx}
\usepackage{indentfirst}


%\setmainfont {NanumMyeongjo}
\setmainfont {UnBatang}
\setmonofont[Scale=0.8]{DejaVu Sans Mono}

\input {../../lststyles.tex}

\begin {document}

\title {Linux 의 task\_struct 크기 조사}
\input {../../reportauthor.tex}
\maketitle

\section {task\_struct}
리눅스 커널의 task\_struct 는 프로세스의 메모리 매핑과 프로그램 카운터, 가동 상태, 자식-부모 관계, ptrace 목록, 쓰레드 목록 등을 가지고 있는 구조체이다. 각 프로세스는 이 구조체에 저장\linebreak
되며, 스케쥴러는 task\_struct 로 이루어진 큐에서 - CFS 의 경우 정확히는 Red-black tree 로 구성된 태스크 ``타임라인'' - 태스크 상태를 되살리고, Time slice 를 소비하면 다시 프로세스를 정지시키고 큐에 넣는 일을 반복한다.
\section {task\_struct 의 크기 구하기}
task\_struct 의 크기를 구하기 위해 커널 모듈을 사용하여 커널 안에서 sizeof(struct task\_struct) 를 실행하고 그 결과를 커널 로그에 나타내도록 한다. 모듈을 사용하는 이유는, 커널 전체 재 컴파일 없이도 모듈을 삽입,제거함으로서 재 부팅 없이 커널에 코드를 로드해서 동작시킬 수 있게 하기 위함이다.
\section {코드}
Makefile
\lstinputlisting [style=customtxt]{makefile-module}
module main
\lstinputlisting [style=customc]{main.c}
dmesg output
\lstinputlisting [style=customtxt]{dmesg.log}
\section {결론}
실제로 구조체의 크기를 측정해본 결과 Linux 2.6.35.6 기준으로 대략 5.76KiB 정도를 차지하는 것을 알 수 있었다.
\end {document}
