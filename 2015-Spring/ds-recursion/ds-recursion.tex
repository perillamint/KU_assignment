%!TEX TS-program = xelatex

\documentclass {article}

\usepackage{xetexko}
\usepackage[a4paper]{geometry}
\usepackage[usenames,dvipsnames]{xcolor}
\usepackage{mathtools}
\usepackage{amsmath}
\usepackage{fontspec}
\usepackage{hyperref}
\usepackage{graphicx}
\usepackage{listings}
\usepackage{makeidx}
\usepackage{indentfirst}
\usepackage{tikz}
\usetikzlibrary{arrows,automata}


%\setmainfont {NanumMyeongjo}
\setmainfont {UnBatang}
\setmonofont[Scale=0.8]{DejaVu Sans Mono}

\lstdefinestyle{diff}{
  belowcaptionskip=1\baselineskip,
  breaklines=true,
  frame=L,
  xleftmargin=\parindent,
  showstringspaces=false,
  % Diffstart
  morecomment=[f][\color{gray}]{@@},
  % Diffincl
  morecomment=[f][\color{Green}]{+},
  % Diffrem
  morecomment=[f][\color{Red}]{-},
  basicstyle=\footnotesize\ttfamily,
}

\lstdefinestyle{customtxt}{
  belowcaptionskip=1\baselineskip,
  breaklines=true,
  frame=L,
  xleftmargin=\parindent,
  showstringspaces=false,
  basicstyle=\footnotesize\ttfamily,
}

\lstdefinestyle{customc}{
  belowcaptionskip=1\baselineskip,
  breaklines=true,
  frame=L,
  xleftmargin=\parindent,
  language=C,
  showstringspaces=false,
  basicstyle=\footnotesize\ttfamily,
  keywordstyle=\bfseries\color{green!40!black},
  commentstyle=\itshape\color{purple!40!black},
  identifierstyle=\color{blue},
  stringstyle=\color{orange},
}

\lstdefinestyle{customrs}{
  belowcaptionskip=1\baselineskip,
  breaklines=true,
  frame=L,
  xleftmargin=\parindent,
  showstringspaces=false,
  morekeywords={fn,let,mut,pub,use,impl,struct,unsafe,if,for},
  morecomment=[l]{//},
  morecomment=[n]{/*}{*/},
  basicstyle=\footnotesize\ttfamily,
  keywordstyle=\bfseries\color{green!40!black},
  commentstyle=\itshape\color{purple!40!black},
  identifierstyle=\color{blue},
  stringstyle=\color{orange},
}


\begin {document}

\title {Recursion 이 존재할 때 알고리즘의 시간복잡도 계산}
\input {../../reportauthor.tex}
\maketitle

\section{서론}
재귀를 사용하는 알고리즘의 시간 복잡도를 점화식을 이용해 일반항을 구함으로서 계산한다.
\section{Factorial}
구현
\lstinputlisting [style=customc]{factorial.c}
시간복잡도 계산
\\\\
이 팩토리얼을 계산하는 함수는 $ n = 0 $ 일때 종료된다. \\
이는
\begin {equation}
  T(0) = O(1)
\end {equation}
\begin {equation}
  T(n) = O(1) + T(n - 1)
\end {equation}
과 같이 시간 복잡도에 대한 점화식을 세울 수 있다. \\
이 점화식은
\begin {align*}
  T(n) &= O(1) + \underbrace { O(1) + T(n-2) }_{ T(n-1) } \\
       &= O(1) + O(1) + \underbrace { O(1) + T(n-3)}_{ T(n-2) }\\
       &= k * O(1) + T(n - k) \\
       &= n * O(1) \\
       &= O(n)
\end {align*}
가 되므로, 재귀호출로 팩토리얼을 계산하는 알고리즘의 시간 복잡도는 $ O(n) $ 임을 알 수 있다.
\section{Hanoi tower}
구현
\lstinputlisting [style=customc]{hanoi.c}
시간복잡도 계산
\\\\
하노이 타워 알고리즘은 남은 고리의 갯수가 0일때 종료된다. \\
이는
\begin {equation}
  T(0) = O(1)
\end {equation}
\begin {equation}
  T(n) = O(1) + 2*T(n-1)
\end {equation}
과 같이 식이 세워진다. \\
이 점화식은
\begin {align*}
  T(n) &= O(1) + 2*T(n-1) \\
       &= O(1) + \underbrace {2*O(1) + 4*T(n-2)}_{T(n-1)} \\
       &= O(1) + 2*O(1) + \underbrace {4*O(1) + 8*T(n-3)}_{T(n-2)} \\
       &= (n^2 - 1) * O(1) \\
       &= O(n^2)
\end {align*}
가 되어 하노이 타워의 시간복잡도는 $ O(n^2) $ 가 된다.

\section {결론}
이와 같이 재귀적 호출이 들어 있는 알고리즘의 시간 복잡도는 정규식을 이용해서, 일반항을 구하면 시간 복잡도를 구할 수 있다.
\end {document}
