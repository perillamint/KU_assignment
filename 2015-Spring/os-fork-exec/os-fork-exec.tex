%!TEX TS-program = xelatex

\documentclass {article}

\usepackage{xetexko}
\usepackage[a4paper]{geometry}
\usepackage[usenames,dvipsnames]{xcolor}
\usepackage{mathtools}
\usepackage{amsmath}
\usepackage{fontspec}
\usepackage{hyperref}
\usepackage{graphicx}
\usepackage{listings}
\usepackage{makeidx}
\usepackage{indentfirst}


%\setmainfont {NanumMyeongjo}
\setmainfont {UnBatang}
\setmonofont[Scale=0.8]{DejaVu Sans Mono}

\input {../../lststyles.tex}

\begin {document}

\title {fork, exec 와 process limit}
\input {../../reportauthor.tex}
\maketitle

\section {fork}
fork 시스템 콜은 프로세스 자기 자신을 복제해서 그대로 서로 다른 프로세스로 만든다. fork() 에 의해 복제된 프로세스는 부모 프로세스와 다음과 같은 것들을 제외하고 같다.

주요 차이점\cite {fork}
\begin {itemize}
  \item 자식 프로세스는 고유한 PID 값을 가진다
  \item 자식 프로세스의 부모 PID 값은 그 부모의 PID 를 따른다
  \item 부모의 memory lock 상태는 자식에서 유지되지 않는다.
  \item 자원 사용 카운터는 초기화된다
  \item 자식에서 부모의 세마포어 변경은 유지되지 않는다
  \item 자식은 부모가 설정한 타이머를 이어받지 않는다.
  \item 자식은 부모가 실행한 비동기 I/O 동작을 이어받지 않는다.
\end {itemize}

이때 프로세스의 메모리는 Copy-on-Write 의 형태로 복사되며, 이는 fork 순간의 오버헤드를 줄이는 역활을 한다.

\section {execve}
execve 시스템 콜은 시스템 콜이 성공할 시 리턴하지 않고, 프로세스의 text, data, bss, stack 섹션을 인자로 주워진 프로그램으로 덮어씌운다.\cite{execve} 이는 fork() 와 함께 다른 프로그램을 자식 프로세스로 실행하는데 사용된다.


\section {fork limit}
시스템이 생성할 수 있는 프로세스의 수에는 한계가 있다. 이 제한은 일반 유저와 root 가 서로 다른데, 이는 일반 유저가 fork bomb 을 이용해 시스템을 동작 불가능하게 막기 위해서이다.
일반 유저는 hard limit 이 적용되어 시스템 전체를 망가뜨리지 못하게 되어 있고, root 계정의 경우 시스템의 한계 (/proc/sys/kernel/pid\_max 파일로 제한) 까지 프로세스를 생성할 수 있다.

Fork 한계 측정 소스 코드:
\lstinputlisting [style=customc]{forkbomb/forkbomb.c}
출력 결과물:
\lstinputlisting [style=customtxt]{forkbomb.log}

\bibliographystyle {plain}
\bibliography {man}
\end {document}
