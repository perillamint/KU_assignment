%!TEX TS-program = xelatex

\documentclass {article}

\usepackage{xetexko}
\usepackage[a4paper]{geometry}
\usepackage[usenames,dvipsnames]{xcolor}
\usepackage{mathtools}
\usepackage{amsmath}
\usepackage{fontspec}
\usepackage{hyperref}
\usepackage{graphicx}
\usepackage{listings}
\usepackage{makeidx}
\usepackage{indentfirst}
\usepackage{tikz}
\usetikzlibrary{arrows,automata}


%\setmainfont {NanumMyeongjo}
\setmainfont {UnBatang}
\setmonofont[Scale=0.8]{DejaVu Sans Mono}

\lstdefinestyle{diff}{
  belowcaptionskip=1\baselineskip,
  breaklines=true,
  frame=L,
  xleftmargin=\parindent,
  showstringspaces=false,
  % Diffstart
  morecomment=[f][\color{gray}]{@@},
  % Diffincl
  morecomment=[f][\color{Green}]{+},
  % Diffrem
  morecomment=[f][\color{Red}]{-},
  basicstyle=\footnotesize\ttfamily,
}

\lstdefinestyle{customtxt}{
  belowcaptionskip=1\baselineskip,
  breaklines=true,
  frame=L,
  xleftmargin=\parindent,
  showstringspaces=false,
  basicstyle=\footnotesize\ttfamily,
}

\lstdefinestyle{customc}{
  belowcaptionskip=1\baselineskip,
  breaklines=true,
  frame=L,
  xleftmargin=\parindent,
  language=C,
  showstringspaces=false,
  basicstyle=\footnotesize\ttfamily,
  keywordstyle=\bfseries\color{green!40!black},
  commentstyle=\itshape\color{purple!40!black},
  identifierstyle=\color{blue},
  stringstyle=\color{orange},
}

\lstdefinestyle{customrs}{
  belowcaptionskip=1\baselineskip,
  breaklines=true,
  frame=L,
  xleftmargin=\parindent,
  showstringspaces=false,
  morekeywords={fn,let,mut,pub,use,impl,struct,unsafe,if,for},
  morecomment=[l]{//},
  morecomment=[n]{/*}{*/},
  basicstyle=\footnotesize\ttfamily,
  keywordstyle=\bfseries\color{green!40!black},
  commentstyle=\itshape\color{purple!40!black},
  identifierstyle=\color{blue},
  stringstyle=\color{orange},
}


\begin {document}

\title {RAM vs. Disk vs. Keyboard \\ 데이터를 전송하는데 걸리는 시간 비교}
\input {../../reportauthor.tex}
\maketitle

\section {서론}
시스템을 이루는 각 장치들의 속도를 비교해 봄으로서 왜 시스템이 유저 입력이 있을 때 멈추지 않는 지를 알아본다.
\section {RAM}
메모리가 데이터를 쓰는 데 걸리는 시간은
\begin {equation}
  \frac {\mathrm{TransferSize}}{\mathrm{Clock} * \mathrm{DataTransferPerClock} * \mathrm{DataBusWidth} * \mathrm{Channel}}
\end {equation}
과 같이 계산할 수 있다.
DDR3-1600 듀얼 채널 구성을 가정하면 전송 최소단위는 128bit 이다.
따라서 한 바이트를 전송하려면, 메모리가 한번 동작해야 하고, 이를 이용해 1byte 를 쓰는 데 걸리\linebreak
는 시간을 계산하면(7byte 낭비)
\begin {equation}
  \frac {128}{200*10^6 * 2 * 64 * 2} \mathrm{second} \approx 2.5\mathrm{nanosecond}
\end {equation}
이 걸린다.
\section {Disk}
OS는 AHCI 를 사용해서 SATA 장치와 데이터를 교환한다. 데이터를 전송할 때에는, Frame Information Structure 를 사용해서 데이터와 명령을 포장해서 보내는데, 이 데이터 구조는
\lstinputlisting [style=customc]{datafis.c}
와 같다.
따라서 1byte 를 보내는 데 드는 데이터 FIS 의 크기는
8byte 가 되고(3Byte 는 낭비), SATA 3.0 의 대역폭은 6Gbit/s\cite{annjansen00} 이므로
\begin {equation}
\frac {8}{6 * 10^9} \mathrm{second} \approx 1.33 \mathrm{nanosecond}
\end {equation}
가 된다.
\section {Keyboard}
키보드는 PS/2 포트를 사용하는 것으로 가정한다.
PS/2 방식의 키보드는 10-16.7 KHz 의 클럭 속도로 i8042 PS/2 컨트롤러와 통신한다.\cite {adamchapweske00} PS/2 컨트롤러는
start bit + 8bit + odd parity + end bit 를 수신한 뒤 데이터를 처리해서 인터럽트를 발생시킨다.
PS/2 를 통해서 한 메시지가 전송되기까\linebreak
지는,
\begin {equation}
\frac{11}{16700} \mathrm{second} \approx 660 \mathrm{microsecond}
\end {equation}
만큼의 시간이 필요하다.
또한, 키보드가 보내는 모든 메시지가 유저 인풋에 대한 정보는 아니기에, 실제 발생하는 IRQ 빈도는 더 낮다고 볼 수 있다.
\section {결론}
유저 입력이 들어오는 키보드의 속도는 다른 장비들에 비해 현저히 낮아 시스템의 동작을 방해할 정도는 아니다. 또한, 현대 CPU 내에는 캐싱 메커니즘이 존재하므로, 메모리 액세스가 동작하는 속도가 더 빨라지는 것과, 큰 데이터를 한 명령에 실어 보낼 수 있는 것을 감안하면, 디스크 I/O 가 SATA 3.0 최대 속도로 동작한다 하더라도 시스템을 방해할 정도는 아니라는 것을 알 수 있었다.

\bibliographystyle {plain}
\bibliography {computer-engineering,sataio}
\end {document}
