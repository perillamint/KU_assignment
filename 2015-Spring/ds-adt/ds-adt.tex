%!TEX TS-program = xelatex

\documentclass {article}

\usepackage{xetexko}
\usepackage[a4paper]{geometry}
\usepackage[usenames,dvipsnames]{xcolor}
\usepackage{mathtools}
\usepackage{amsmath}
\usepackage{fontspec}
\usepackage{hyperref}
\usepackage{graphicx}
\usepackage{listings}
\usepackage{makeidx}
\usepackage{indentfirst}
\usepackage{tikz}
\usetikzlibrary{arrows,automata}


%\setmainfont {NanumMyeongjo}
\setmainfont {UnBatang}
\setmonofont[Scale=0.8]{DejaVu Sans Mono}

\lstdefinestyle{diff}{
  belowcaptionskip=1\baselineskip,
  breaklines=true,
  frame=L,
  xleftmargin=\parindent,
  showstringspaces=false,
  % Diffstart
  morecomment=[f][\color{gray}]{@@},
  % Diffincl
  morecomment=[f][\color{Green}]{+},
  % Diffrem
  morecomment=[f][\color{Red}]{-},
  basicstyle=\footnotesize\ttfamily,
}

\lstdefinestyle{customtxt}{
  belowcaptionskip=1\baselineskip,
  breaklines=true,
  frame=L,
  xleftmargin=\parindent,
  showstringspaces=false,
  basicstyle=\footnotesize\ttfamily,
}

\lstdefinestyle{customc}{
  belowcaptionskip=1\baselineskip,
  breaklines=true,
  frame=L,
  xleftmargin=\parindent,
  language=C,
  showstringspaces=false,
  basicstyle=\footnotesize\ttfamily,
  keywordstyle=\bfseries\color{green!40!black},
  commentstyle=\itshape\color{purple!40!black},
  identifierstyle=\color{blue},
  stringstyle=\color{orange},
}

\lstdefinestyle{customrs}{
  belowcaptionskip=1\baselineskip,
  breaklines=true,
  frame=L,
  xleftmargin=\parindent,
  showstringspaces=false,
  morekeywords={fn,let,mut,pub,use,impl,struct,unsafe,if,for},
  morecomment=[l]{//},
  morecomment=[n]{/*}{*/},
  basicstyle=\footnotesize\ttfamily,
  keywordstyle=\bfseries\color{green!40!black},
  commentstyle=\itshape\color{purple!40!black},
  identifierstyle=\color{blue},
  stringstyle=\color{orange},
}


\begin {document}

\title {DT, ADT, Object 의 차이}
\input {../../reportauthor.tex}
\maketitle

\section {Data Type}
\(~~\)Data type: 컴퓨터의 Linear 한 메모리를 인간이 보다 쉽게 사용할 수 있게 하기 위해, 메모리 영역을 추상화하여 보다 쉽게 값을 쓸 수 있도록 추상화 한 구조.
\vspace{\baselineskip}

코드 구현 - C:
\lstinputlisting [style=customc]{ds-adt-dt.c}
\vspace{\baselineskip}
코드 출력:
\lstinputlisting [style=customtxt]{ds-adt-dt.txt}

\section {Abstracted Data Type}
\(~~\)Abstracted Data Type: Data Type 들을 한번 더 추상화시켜, 자료를 다루는 방법을 새롭게 정의한 것.
ADT에서는, ADT 가 추상화시킨 내부의 데이터에 직접적으로 접근할 수 없고, 추상화된 접근을 정의한 ADT 의 API를 통해서만 접근이 가능하며, 이 ADT의 API는 ADT내부에 어떤 형식으로\linebreak
데이터가 저장되는지를 알지 못해도 ADT 내부의 데이터에 ``간접적으로'' 접근할 수 있도록 한다.

\vspace{\baselineskip}
코드 구현 - C:
\lstinputlisting [style=customc]{ds-adt-adt.c}
\vspace{\baselineskip}
코드 출력:
\lstinputlisting [style=customtxt]{ds-adt-adt.txt}
\vspace{\baselineskip}
\section {Object}
\(~~\)Object: Abstract Data Type (혹은 클래스) 에 정의된 것을 실제로 메모리 위에 올려서, 데이터를 기록할 수 있는 ``상태'' 를 가질 수 있게 된 것. 각각의 객체는 자신의 상태를 가지며, 이 상태는 객체를 찍어낸 ADT 에 (혹은 클래스) 정의된 API를 통해 바뀐다.
% 치킨먹고싶다
\vspace{\baselineskip}
코드 구현 - CPP:
\lstinputlisting [style=customc]{ds-adt-obj.cpp}
코드 출력:
\lstinputlisting [style=customtxt]{ds-adt-obj.txt}
\end {document}
