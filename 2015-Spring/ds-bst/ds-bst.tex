%!TEX TS-program = xelatex

\documentclass {article}

\usepackage{xetexko}
\usepackage[a4paper]{geometry}
\usepackage[usenames,dvipsnames]{xcolor}
\usepackage{mathtools}
\usepackage{amsmath}
\usepackage{fontspec}
\usepackage{hyperref}
\usepackage{graphicx}
\usepackage{listings}
\usepackage{makeidx}
\usepackage{indentfirst}


%\setmainfont {NanumMyeongjo}
\setmainfont {UnBatang}
\setmonofont[Scale=0.8]{DejaVu Sans Mono}

\input {../../lststyles.tex}

\begin {document}

\title {BST와 Linked list 의 실행 시간 비교}
\input {../../reportauthor.tex}
\maketitle

\section {Insert operation}
BST 의 경우 Insert 동작 실행시, BST 의 조건에 맞는 노드를 찾아 내려가게 된다. 이는 탐색 과정과 같으며, 이는 평균적으로 $ O(\log(n)) $ 의 시간 복잡도가 소요된다. 반면, Linked list 의 경우, 단지 리스트에 해당 노드를 추가하면 되기에, 추가에는 $ O(1) $ 의 상수 시간복잡도가 소요된다.

하지만, 데이터가 정렬되어 있을 경우, BST에서 검색 동작은, Linked list 의 그것과 같게 되어, 시간복잡도가 $ O(n) $ 으로 느려지게 된다.

\section {Search operation}
BST의 경우, 트리 자체로 2진 탐색을 할 수 있다. 이는 이진 트리가 데이터 탐색을 하는 데 있어, Linked list 보다 일반적으로 더 빠른 성능을 낼 수 있음을 의미한다. 이진 트리는 트리가 Balanced 에 가까운 일반적인 상태에서는 $ O(\log(n)) $ 의 시간복잡도를 가진다.

반면, Linked list 는, 리스트의 앞부터 끝까지 탐색을 진행하면서, 해당 데이터가 있는지를 확인하는 방식이므로, 검색에 있어, $ O(n) $ 의 시간복잡도를 가지게 된다.

무작위 데이터를 이용한 측정은 다음과 같았다.
\lstinputlisting [style=customtxt]{benchmark.dat}

이를 그래프로 그리고 Fitting 해 보면, 두 경우 모두, 각 자료구조의 시간복잡도에 맞는 것을 알 수 있다. 

\begin {figure}
  \centering
  % GNUPLOT: LaTeX picture
\setlength{\unitlength}{0.240900pt}
\ifx\plotpoint\undefined\newsavebox{\plotpoint}\fi
\begin{picture}(1500,900)(0,0)
\sbox{\plotpoint}{\rule[-0.200pt]{0.400pt}{0.400pt}}%
\put(211.0,131.0){\rule[-0.200pt]{4.818pt}{0.400pt}}
\put(191,131){\makebox(0,0)[r]{$0$}}
\put(1419.0,131.0){\rule[-0.200pt]{4.818pt}{0.400pt}}
\put(211.0,204.0){\rule[-0.200pt]{4.818pt}{0.400pt}}
\put(191,204){\makebox(0,0)[r]{$20000$}}
\put(1419.0,204.0){\rule[-0.200pt]{4.818pt}{0.400pt}}
\put(211.0,277.0){\rule[-0.200pt]{4.818pt}{0.400pt}}
\put(191,277){\makebox(0,0)[r]{$40000$}}
\put(1419.0,277.0){\rule[-0.200pt]{4.818pt}{0.400pt}}
\put(211.0,349.0){\rule[-0.200pt]{4.818pt}{0.400pt}}
\put(191,349){\makebox(0,0)[r]{$60000$}}
\put(1419.0,349.0){\rule[-0.200pt]{4.818pt}{0.400pt}}
\put(211.0,422.0){\rule[-0.200pt]{4.818pt}{0.400pt}}
\put(191,422){\makebox(0,0)[r]{$80000$}}
\put(1419.0,422.0){\rule[-0.200pt]{4.818pt}{0.400pt}}
\put(211.0,495.0){\rule[-0.200pt]{4.818pt}{0.400pt}}
\put(191,495){\makebox(0,0)[r]{$100000$}}
\put(1419.0,495.0){\rule[-0.200pt]{4.818pt}{0.400pt}}
\put(211.0,568.0){\rule[-0.200pt]{4.818pt}{0.400pt}}
\put(191,568){\makebox(0,0)[r]{$120000$}}
\put(1419.0,568.0){\rule[-0.200pt]{4.818pt}{0.400pt}}
\put(211.0,641.0){\rule[-0.200pt]{4.818pt}{0.400pt}}
\put(191,641){\makebox(0,0)[r]{$140000$}}
\put(1419.0,641.0){\rule[-0.200pt]{4.818pt}{0.400pt}}
\put(211.0,713.0){\rule[-0.200pt]{4.818pt}{0.400pt}}
\put(191,713){\makebox(0,0)[r]{$160000$}}
\put(1419.0,713.0){\rule[-0.200pt]{4.818pt}{0.400pt}}
\put(211.0,786.0){\rule[-0.200pt]{4.818pt}{0.400pt}}
\put(191,786){\makebox(0,0)[r]{$180000$}}
\put(1419.0,786.0){\rule[-0.200pt]{4.818pt}{0.400pt}}
\put(211.0,859.0){\rule[-0.200pt]{4.818pt}{0.400pt}}
\put(191,859){\makebox(0,0)[r]{$200000$}}
\put(1419.0,859.0){\rule[-0.200pt]{4.818pt}{0.400pt}}
\put(211.0,131.0){\rule[-0.200pt]{0.400pt}{4.818pt}}
\put(211,90){\makebox(0,0){$2000$}}
\put(211.0,839.0){\rule[-0.200pt]{0.400pt}{4.818pt}}
\put(365.0,131.0){\rule[-0.200pt]{0.400pt}{4.818pt}}
\put(365,90){\makebox(0,0){$4000$}}
\put(365.0,839.0){\rule[-0.200pt]{0.400pt}{4.818pt}}
\put(518.0,131.0){\rule[-0.200pt]{0.400pt}{4.818pt}}
\put(518,90){\makebox(0,0){$6000$}}
\put(518.0,839.0){\rule[-0.200pt]{0.400pt}{4.818pt}}
\put(672.0,131.0){\rule[-0.200pt]{0.400pt}{4.818pt}}
\put(672,90){\makebox(0,0){$8000$}}
\put(672.0,839.0){\rule[-0.200pt]{0.400pt}{4.818pt}}
\put(825.0,131.0){\rule[-0.200pt]{0.400pt}{4.818pt}}
\put(825,90){\makebox(0,0){$10000$}}
\put(825.0,839.0){\rule[-0.200pt]{0.400pt}{4.818pt}}
\put(979.0,131.0){\rule[-0.200pt]{0.400pt}{4.818pt}}
\put(979,90){\makebox(0,0){$12000$}}
\put(979.0,839.0){\rule[-0.200pt]{0.400pt}{4.818pt}}
\put(1132.0,131.0){\rule[-0.200pt]{0.400pt}{4.818pt}}
\put(1132,90){\makebox(0,0){$14000$}}
\put(1132.0,839.0){\rule[-0.200pt]{0.400pt}{4.818pt}}
\put(1286.0,131.0){\rule[-0.200pt]{0.400pt}{4.818pt}}
\put(1286,90){\makebox(0,0){$16000$}}
\put(1286.0,839.0){\rule[-0.200pt]{0.400pt}{4.818pt}}
\put(1439.0,131.0){\rule[-0.200pt]{0.400pt}{4.818pt}}
\put(1439,90){\makebox(0,0){$18000$}}
\put(1439.0,839.0){\rule[-0.200pt]{0.400pt}{4.818pt}}
\put(211.0,131.0){\rule[-0.200pt]{0.400pt}{175.375pt}}
\put(211.0,131.0){\rule[-0.200pt]{295.825pt}{0.400pt}}
\put(1439.0,131.0){\rule[-0.200pt]{0.400pt}{175.375pt}}
\put(211.0,859.0){\rule[-0.200pt]{295.825pt}{0.400pt}}
\put(30,495){\makebox(0,0){\shortstack{Time\\ (nsec)}}}
\put(825,29){\makebox(0,0){Data set size}}
\put(1279,818){\makebox(0,0)[r]{'linkedlist.dat' index 0}}
\put(215,184){\makebox(0,0){$\bullet$}}
\put(372,246){\makebox(0,0){$\bullet$}}
\put(529,496){\makebox(0,0){$\bullet$}}
\put(686,508){\makebox(0,0){$\bullet$}}
\put(843,549){\makebox(0,0){$\bullet$}}
\put(1001,640){\makebox(0,0){$\bullet$}}
\put(1158,770){\makebox(0,0){$\bullet$}}
\put(1315,812){\makebox(0,0){$\bullet$}}
\put(1349,818){\makebox(0,0){$\bullet$}}
\put(1279,777){\makebox(0,0)[r]{11.8038 * x + 0.466543}}
\put(1299.0,777.0){\rule[-0.200pt]{24.090pt}{0.400pt}}
\put(215,219){\usebox{\plotpoint}}
\multiput(215.00,219.59)(0.943,0.482){9}{\rule{0.833pt}{0.116pt}}
\multiput(215.00,218.17)(9.270,6.000){2}{\rule{0.417pt}{0.400pt}}
\multiput(226.00,225.59)(0.943,0.482){9}{\rule{0.833pt}{0.116pt}}
\multiput(226.00,224.17)(9.270,6.000){2}{\rule{0.417pt}{0.400pt}}
\multiput(237.00,231.59)(0.798,0.485){11}{\rule{0.729pt}{0.117pt}}
\multiput(237.00,230.17)(9.488,7.000){2}{\rule{0.364pt}{0.400pt}}
\multiput(248.00,238.59)(0.943,0.482){9}{\rule{0.833pt}{0.116pt}}
\multiput(248.00,237.17)(9.270,6.000){2}{\rule{0.417pt}{0.400pt}}
\multiput(259.00,244.59)(0.943,0.482){9}{\rule{0.833pt}{0.116pt}}
\multiput(259.00,243.17)(9.270,6.000){2}{\rule{0.417pt}{0.400pt}}
\multiput(270.00,250.59)(0.943,0.482){9}{\rule{0.833pt}{0.116pt}}
\multiput(270.00,249.17)(9.270,6.000){2}{\rule{0.417pt}{0.400pt}}
\multiput(281.00,256.59)(0.798,0.485){11}{\rule{0.729pt}{0.117pt}}
\multiput(281.00,255.17)(9.488,7.000){2}{\rule{0.364pt}{0.400pt}}
\multiput(292.00,263.59)(1.033,0.482){9}{\rule{0.900pt}{0.116pt}}
\multiput(292.00,262.17)(10.132,6.000){2}{\rule{0.450pt}{0.400pt}}
\multiput(304.00,269.59)(0.943,0.482){9}{\rule{0.833pt}{0.116pt}}
\multiput(304.00,268.17)(9.270,6.000){2}{\rule{0.417pt}{0.400pt}}
\multiput(315.00,275.59)(0.943,0.482){9}{\rule{0.833pt}{0.116pt}}
\multiput(315.00,274.17)(9.270,6.000){2}{\rule{0.417pt}{0.400pt}}
\multiput(326.00,281.59)(0.943,0.482){9}{\rule{0.833pt}{0.116pt}}
\multiput(326.00,280.17)(9.270,6.000){2}{\rule{0.417pt}{0.400pt}}
\multiput(337.00,287.59)(0.798,0.485){11}{\rule{0.729pt}{0.117pt}}
\multiput(337.00,286.17)(9.488,7.000){2}{\rule{0.364pt}{0.400pt}}
\multiput(348.00,294.59)(0.943,0.482){9}{\rule{0.833pt}{0.116pt}}
\multiput(348.00,293.17)(9.270,6.000){2}{\rule{0.417pt}{0.400pt}}
\multiput(359.00,300.59)(0.943,0.482){9}{\rule{0.833pt}{0.116pt}}
\multiput(359.00,299.17)(9.270,6.000){2}{\rule{0.417pt}{0.400pt}}
\multiput(370.00,306.59)(0.943,0.482){9}{\rule{0.833pt}{0.116pt}}
\multiput(370.00,305.17)(9.270,6.000){2}{\rule{0.417pt}{0.400pt}}
\multiput(381.00,312.59)(0.874,0.485){11}{\rule{0.786pt}{0.117pt}}
\multiput(381.00,311.17)(10.369,7.000){2}{\rule{0.393pt}{0.400pt}}
\multiput(393.00,319.59)(0.943,0.482){9}{\rule{0.833pt}{0.116pt}}
\multiput(393.00,318.17)(9.270,6.000){2}{\rule{0.417pt}{0.400pt}}
\multiput(404.00,325.59)(0.943,0.482){9}{\rule{0.833pt}{0.116pt}}
\multiput(404.00,324.17)(9.270,6.000){2}{\rule{0.417pt}{0.400pt}}
\multiput(415.00,331.59)(0.943,0.482){9}{\rule{0.833pt}{0.116pt}}
\multiput(415.00,330.17)(9.270,6.000){2}{\rule{0.417pt}{0.400pt}}
\multiput(426.00,337.59)(0.943,0.482){9}{\rule{0.833pt}{0.116pt}}
\multiput(426.00,336.17)(9.270,6.000){2}{\rule{0.417pt}{0.400pt}}
\multiput(437.00,343.59)(0.798,0.485){11}{\rule{0.729pt}{0.117pt}}
\multiput(437.00,342.17)(9.488,7.000){2}{\rule{0.364pt}{0.400pt}}
\multiput(448.00,350.59)(0.943,0.482){9}{\rule{0.833pt}{0.116pt}}
\multiput(448.00,349.17)(9.270,6.000){2}{\rule{0.417pt}{0.400pt}}
\multiput(459.00,356.59)(0.943,0.482){9}{\rule{0.833pt}{0.116pt}}
\multiput(459.00,355.17)(9.270,6.000){2}{\rule{0.417pt}{0.400pt}}
\multiput(470.00,362.59)(0.943,0.482){9}{\rule{0.833pt}{0.116pt}}
\multiput(470.00,361.17)(9.270,6.000){2}{\rule{0.417pt}{0.400pt}}
\multiput(481.00,368.59)(0.874,0.485){11}{\rule{0.786pt}{0.117pt}}
\multiput(481.00,367.17)(10.369,7.000){2}{\rule{0.393pt}{0.400pt}}
\multiput(493.00,375.59)(0.943,0.482){9}{\rule{0.833pt}{0.116pt}}
\multiput(493.00,374.17)(9.270,6.000){2}{\rule{0.417pt}{0.400pt}}
\multiput(504.00,381.59)(0.943,0.482){9}{\rule{0.833pt}{0.116pt}}
\multiput(504.00,380.17)(9.270,6.000){2}{\rule{0.417pt}{0.400pt}}
\multiput(515.00,387.59)(0.943,0.482){9}{\rule{0.833pt}{0.116pt}}
\multiput(515.00,386.17)(9.270,6.000){2}{\rule{0.417pt}{0.400pt}}
\multiput(526.00,393.59)(0.943,0.482){9}{\rule{0.833pt}{0.116pt}}
\multiput(526.00,392.17)(9.270,6.000){2}{\rule{0.417pt}{0.400pt}}
\multiput(537.00,399.59)(0.798,0.485){11}{\rule{0.729pt}{0.117pt}}
\multiput(537.00,398.17)(9.488,7.000){2}{\rule{0.364pt}{0.400pt}}
\multiput(548.00,406.59)(0.943,0.482){9}{\rule{0.833pt}{0.116pt}}
\multiput(548.00,405.17)(9.270,6.000){2}{\rule{0.417pt}{0.400pt}}
\multiput(559.00,412.59)(0.943,0.482){9}{\rule{0.833pt}{0.116pt}}
\multiput(559.00,411.17)(9.270,6.000){2}{\rule{0.417pt}{0.400pt}}
\multiput(570.00,418.59)(0.943,0.482){9}{\rule{0.833pt}{0.116pt}}
\multiput(570.00,417.17)(9.270,6.000){2}{\rule{0.417pt}{0.400pt}}
\multiput(581.00,424.59)(0.874,0.485){11}{\rule{0.786pt}{0.117pt}}
\multiput(581.00,423.17)(10.369,7.000){2}{\rule{0.393pt}{0.400pt}}
\multiput(593.00,431.59)(0.943,0.482){9}{\rule{0.833pt}{0.116pt}}
\multiput(593.00,430.17)(9.270,6.000){2}{\rule{0.417pt}{0.400pt}}
\multiput(604.00,437.59)(0.943,0.482){9}{\rule{0.833pt}{0.116pt}}
\multiput(604.00,436.17)(9.270,6.000){2}{\rule{0.417pt}{0.400pt}}
\multiput(615.00,443.59)(0.943,0.482){9}{\rule{0.833pt}{0.116pt}}
\multiput(615.00,442.17)(9.270,6.000){2}{\rule{0.417pt}{0.400pt}}
\multiput(626.00,449.59)(0.943,0.482){9}{\rule{0.833pt}{0.116pt}}
\multiput(626.00,448.17)(9.270,6.000){2}{\rule{0.417pt}{0.400pt}}
\multiput(637.00,455.59)(0.798,0.485){11}{\rule{0.729pt}{0.117pt}}
\multiput(637.00,454.17)(9.488,7.000){2}{\rule{0.364pt}{0.400pt}}
\multiput(648.00,462.59)(0.943,0.482){9}{\rule{0.833pt}{0.116pt}}
\multiput(648.00,461.17)(9.270,6.000){2}{\rule{0.417pt}{0.400pt}}
\multiput(659.00,468.59)(0.943,0.482){9}{\rule{0.833pt}{0.116pt}}
\multiput(659.00,467.17)(9.270,6.000){2}{\rule{0.417pt}{0.400pt}}
\multiput(670.00,474.59)(0.943,0.482){9}{\rule{0.833pt}{0.116pt}}
\multiput(670.00,473.17)(9.270,6.000){2}{\rule{0.417pt}{0.400pt}}
\multiput(681.00,480.59)(0.874,0.485){11}{\rule{0.786pt}{0.117pt}}
\multiput(681.00,479.17)(10.369,7.000){2}{\rule{0.393pt}{0.400pt}}
\multiput(693.00,487.59)(0.943,0.482){9}{\rule{0.833pt}{0.116pt}}
\multiput(693.00,486.17)(9.270,6.000){2}{\rule{0.417pt}{0.400pt}}
\multiput(704.00,493.59)(0.943,0.482){9}{\rule{0.833pt}{0.116pt}}
\multiput(704.00,492.17)(9.270,6.000){2}{\rule{0.417pt}{0.400pt}}
\multiput(715.00,499.59)(0.943,0.482){9}{\rule{0.833pt}{0.116pt}}
\multiput(715.00,498.17)(9.270,6.000){2}{\rule{0.417pt}{0.400pt}}
\multiput(726.00,505.59)(0.943,0.482){9}{\rule{0.833pt}{0.116pt}}
\multiput(726.00,504.17)(9.270,6.000){2}{\rule{0.417pt}{0.400pt}}
\multiput(737.00,511.59)(0.798,0.485){11}{\rule{0.729pt}{0.117pt}}
\multiput(737.00,510.17)(9.488,7.000){2}{\rule{0.364pt}{0.400pt}}
\multiput(748.00,518.59)(0.943,0.482){9}{\rule{0.833pt}{0.116pt}}
\multiput(748.00,517.17)(9.270,6.000){2}{\rule{0.417pt}{0.400pt}}
\multiput(759.00,524.59)(0.943,0.482){9}{\rule{0.833pt}{0.116pt}}
\multiput(759.00,523.17)(9.270,6.000){2}{\rule{0.417pt}{0.400pt}}
\multiput(770.00,530.59)(0.943,0.482){9}{\rule{0.833pt}{0.116pt}}
\multiput(770.00,529.17)(9.270,6.000){2}{\rule{0.417pt}{0.400pt}}
\multiput(781.00,536.59)(0.874,0.485){11}{\rule{0.786pt}{0.117pt}}
\multiput(781.00,535.17)(10.369,7.000){2}{\rule{0.393pt}{0.400pt}}
\multiput(793.00,543.59)(0.943,0.482){9}{\rule{0.833pt}{0.116pt}}
\multiput(793.00,542.17)(9.270,6.000){2}{\rule{0.417pt}{0.400pt}}
\multiput(804.00,549.59)(0.943,0.482){9}{\rule{0.833pt}{0.116pt}}
\multiput(804.00,548.17)(9.270,6.000){2}{\rule{0.417pt}{0.400pt}}
\multiput(815.00,555.59)(0.943,0.482){9}{\rule{0.833pt}{0.116pt}}
\multiput(815.00,554.17)(9.270,6.000){2}{\rule{0.417pt}{0.400pt}}
\multiput(826.00,561.59)(0.943,0.482){9}{\rule{0.833pt}{0.116pt}}
\multiput(826.00,560.17)(9.270,6.000){2}{\rule{0.417pt}{0.400pt}}
\multiput(837.00,567.59)(0.798,0.485){11}{\rule{0.729pt}{0.117pt}}
\multiput(837.00,566.17)(9.488,7.000){2}{\rule{0.364pt}{0.400pt}}
\multiput(848.00,574.59)(0.943,0.482){9}{\rule{0.833pt}{0.116pt}}
\multiput(848.00,573.17)(9.270,6.000){2}{\rule{0.417pt}{0.400pt}}
\multiput(859.00,580.59)(0.943,0.482){9}{\rule{0.833pt}{0.116pt}}
\multiput(859.00,579.17)(9.270,6.000){2}{\rule{0.417pt}{0.400pt}}
\multiput(870.00,586.59)(1.033,0.482){9}{\rule{0.900pt}{0.116pt}}
\multiput(870.00,585.17)(10.132,6.000){2}{\rule{0.450pt}{0.400pt}}
\multiput(882.00,592.59)(0.798,0.485){11}{\rule{0.729pt}{0.117pt}}
\multiput(882.00,591.17)(9.488,7.000){2}{\rule{0.364pt}{0.400pt}}
\multiput(893.00,599.59)(0.943,0.482){9}{\rule{0.833pt}{0.116pt}}
\multiput(893.00,598.17)(9.270,6.000){2}{\rule{0.417pt}{0.400pt}}
\multiput(904.00,605.59)(0.943,0.482){9}{\rule{0.833pt}{0.116pt}}
\multiput(904.00,604.17)(9.270,6.000){2}{\rule{0.417pt}{0.400pt}}
\multiput(915.00,611.59)(0.943,0.482){9}{\rule{0.833pt}{0.116pt}}
\multiput(915.00,610.17)(9.270,6.000){2}{\rule{0.417pt}{0.400pt}}
\multiput(926.00,617.59)(0.943,0.482){9}{\rule{0.833pt}{0.116pt}}
\multiput(926.00,616.17)(9.270,6.000){2}{\rule{0.417pt}{0.400pt}}
\multiput(937.00,623.59)(0.798,0.485){11}{\rule{0.729pt}{0.117pt}}
\multiput(937.00,622.17)(9.488,7.000){2}{\rule{0.364pt}{0.400pt}}
\multiput(948.00,630.59)(0.943,0.482){9}{\rule{0.833pt}{0.116pt}}
\multiput(948.00,629.17)(9.270,6.000){2}{\rule{0.417pt}{0.400pt}}
\multiput(959.00,636.59)(0.943,0.482){9}{\rule{0.833pt}{0.116pt}}
\multiput(959.00,635.17)(9.270,6.000){2}{\rule{0.417pt}{0.400pt}}
\multiput(970.00,642.59)(1.033,0.482){9}{\rule{0.900pt}{0.116pt}}
\multiput(970.00,641.17)(10.132,6.000){2}{\rule{0.450pt}{0.400pt}}
\multiput(982.00,648.59)(0.798,0.485){11}{\rule{0.729pt}{0.117pt}}
\multiput(982.00,647.17)(9.488,7.000){2}{\rule{0.364pt}{0.400pt}}
\multiput(993.00,655.59)(0.943,0.482){9}{\rule{0.833pt}{0.116pt}}
\multiput(993.00,654.17)(9.270,6.000){2}{\rule{0.417pt}{0.400pt}}
\multiput(1004.00,661.59)(0.943,0.482){9}{\rule{0.833pt}{0.116pt}}
\multiput(1004.00,660.17)(9.270,6.000){2}{\rule{0.417pt}{0.400pt}}
\multiput(1015.00,667.59)(0.943,0.482){9}{\rule{0.833pt}{0.116pt}}
\multiput(1015.00,666.17)(9.270,6.000){2}{\rule{0.417pt}{0.400pt}}
\multiput(1026.00,673.59)(0.943,0.482){9}{\rule{0.833pt}{0.116pt}}
\multiput(1026.00,672.17)(9.270,6.000){2}{\rule{0.417pt}{0.400pt}}
\multiput(1037.00,679.59)(0.798,0.485){11}{\rule{0.729pt}{0.117pt}}
\multiput(1037.00,678.17)(9.488,7.000){2}{\rule{0.364pt}{0.400pt}}
\multiput(1048.00,686.59)(0.943,0.482){9}{\rule{0.833pt}{0.116pt}}
\multiput(1048.00,685.17)(9.270,6.000){2}{\rule{0.417pt}{0.400pt}}
\multiput(1059.00,692.59)(0.943,0.482){9}{\rule{0.833pt}{0.116pt}}
\multiput(1059.00,691.17)(9.270,6.000){2}{\rule{0.417pt}{0.400pt}}
\multiput(1070.00,698.59)(1.033,0.482){9}{\rule{0.900pt}{0.116pt}}
\multiput(1070.00,697.17)(10.132,6.000){2}{\rule{0.450pt}{0.400pt}}
\multiput(1082.00,704.59)(0.798,0.485){11}{\rule{0.729pt}{0.117pt}}
\multiput(1082.00,703.17)(9.488,7.000){2}{\rule{0.364pt}{0.400pt}}
\multiput(1093.00,711.59)(0.943,0.482){9}{\rule{0.833pt}{0.116pt}}
\multiput(1093.00,710.17)(9.270,6.000){2}{\rule{0.417pt}{0.400pt}}
\multiput(1104.00,717.59)(0.943,0.482){9}{\rule{0.833pt}{0.116pt}}
\multiput(1104.00,716.17)(9.270,6.000){2}{\rule{0.417pt}{0.400pt}}
\multiput(1115.00,723.59)(0.943,0.482){9}{\rule{0.833pt}{0.116pt}}
\multiput(1115.00,722.17)(9.270,6.000){2}{\rule{0.417pt}{0.400pt}}
\multiput(1126.00,729.59)(0.943,0.482){9}{\rule{0.833pt}{0.116pt}}
\multiput(1126.00,728.17)(9.270,6.000){2}{\rule{0.417pt}{0.400pt}}
\multiput(1137.00,735.59)(0.798,0.485){11}{\rule{0.729pt}{0.117pt}}
\multiput(1137.00,734.17)(9.488,7.000){2}{\rule{0.364pt}{0.400pt}}
\multiput(1148.00,742.59)(0.943,0.482){9}{\rule{0.833pt}{0.116pt}}
\multiput(1148.00,741.17)(9.270,6.000){2}{\rule{0.417pt}{0.400pt}}
\multiput(1159.00,748.59)(0.943,0.482){9}{\rule{0.833pt}{0.116pt}}
\multiput(1159.00,747.17)(9.270,6.000){2}{\rule{0.417pt}{0.400pt}}
\multiput(1170.00,754.59)(1.033,0.482){9}{\rule{0.900pt}{0.116pt}}
\multiput(1170.00,753.17)(10.132,6.000){2}{\rule{0.450pt}{0.400pt}}
\multiput(1182.00,760.59)(0.798,0.485){11}{\rule{0.729pt}{0.117pt}}
\multiput(1182.00,759.17)(9.488,7.000){2}{\rule{0.364pt}{0.400pt}}
\multiput(1193.00,767.59)(0.943,0.482){9}{\rule{0.833pt}{0.116pt}}
\multiput(1193.00,766.17)(9.270,6.000){2}{\rule{0.417pt}{0.400pt}}
\multiput(1204.00,773.59)(0.943,0.482){9}{\rule{0.833pt}{0.116pt}}
\multiput(1204.00,772.17)(9.270,6.000){2}{\rule{0.417pt}{0.400pt}}
\multiput(1215.00,779.59)(0.943,0.482){9}{\rule{0.833pt}{0.116pt}}
\multiput(1215.00,778.17)(9.270,6.000){2}{\rule{0.417pt}{0.400pt}}
\multiput(1226.00,785.59)(0.943,0.482){9}{\rule{0.833pt}{0.116pt}}
\multiput(1226.00,784.17)(9.270,6.000){2}{\rule{0.417pt}{0.400pt}}
\multiput(1237.00,791.59)(0.798,0.485){11}{\rule{0.729pt}{0.117pt}}
\multiput(1237.00,790.17)(9.488,7.000){2}{\rule{0.364pt}{0.400pt}}
\multiput(1248.00,798.59)(0.943,0.482){9}{\rule{0.833pt}{0.116pt}}
\multiput(1248.00,797.17)(9.270,6.000){2}{\rule{0.417pt}{0.400pt}}
\multiput(1259.00,804.59)(1.033,0.482){9}{\rule{0.900pt}{0.116pt}}
\multiput(1259.00,803.17)(10.132,6.000){2}{\rule{0.450pt}{0.400pt}}
\multiput(1271.00,810.59)(0.943,0.482){9}{\rule{0.833pt}{0.116pt}}
\multiput(1271.00,809.17)(9.270,6.000){2}{\rule{0.417pt}{0.400pt}}
\multiput(1282.00,816.59)(0.798,0.485){11}{\rule{0.729pt}{0.117pt}}
\multiput(1282.00,815.17)(9.488,7.000){2}{\rule{0.364pt}{0.400pt}}
\multiput(1293.00,823.59)(0.943,0.482){9}{\rule{0.833pt}{0.116pt}}
\multiput(1293.00,822.17)(9.270,6.000){2}{\rule{0.417pt}{0.400pt}}
\multiput(1304.00,829.59)(0.943,0.482){9}{\rule{0.833pt}{0.116pt}}
\multiput(1304.00,828.17)(9.270,6.000){2}{\rule{0.417pt}{0.400pt}}
\put(211.0,131.0){\rule[-0.200pt]{0.400pt}{175.375pt}}
\put(211.0,131.0){\rule[-0.200pt]{295.825pt}{0.400pt}}
\put(1439.0,131.0){\rule[-0.200pt]{0.400pt}{175.375pt}}
\put(211.0,859.0){\rule[-0.200pt]{295.825pt}{0.400pt}}
\end{picture}

  \caption {Linked list 의 검색 동작 시간}
  \label{fig:linkedlist}
\end {figure}

\begin {figure}
  \centering
  % GNUPLOT: LaTeX picture
\setlength{\unitlength}{0.240900pt}
\ifx\plotpoint\undefined\newsavebox{\plotpoint}\fi
\begin{picture}(1500,900)(0,0)
\sbox{\plotpoint}{\rule[-0.200pt]{0.400pt}{0.400pt}}%
\put(151.0,131.0){\rule[-0.200pt]{4.818pt}{0.400pt}}
\put(131,131){\makebox(0,0)[r]{$220$}}
\put(1419.0,131.0){\rule[-0.200pt]{4.818pt}{0.400pt}}
\put(151.0,235.0){\rule[-0.200pt]{4.818pt}{0.400pt}}
\put(131,235){\makebox(0,0)[r]{$240$}}
\put(1419.0,235.0){\rule[-0.200pt]{4.818pt}{0.400pt}}
\put(151.0,339.0){\rule[-0.200pt]{4.818pt}{0.400pt}}
\put(131,339){\makebox(0,0)[r]{$260$}}
\put(1419.0,339.0){\rule[-0.200pt]{4.818pt}{0.400pt}}
\put(151.0,443.0){\rule[-0.200pt]{4.818pt}{0.400pt}}
\put(131,443){\makebox(0,0)[r]{$280$}}
\put(1419.0,443.0){\rule[-0.200pt]{4.818pt}{0.400pt}}
\put(151.0,547.0){\rule[-0.200pt]{4.818pt}{0.400pt}}
\put(131,547){\makebox(0,0)[r]{$300$}}
\put(1419.0,547.0){\rule[-0.200pt]{4.818pt}{0.400pt}}
\put(151.0,651.0){\rule[-0.200pt]{4.818pt}{0.400pt}}
\put(131,651){\makebox(0,0)[r]{$320$}}
\put(1419.0,651.0){\rule[-0.200pt]{4.818pt}{0.400pt}}
\put(151.0,755.0){\rule[-0.200pt]{4.818pt}{0.400pt}}
\put(131,755){\makebox(0,0)[r]{$340$}}
\put(1419.0,755.0){\rule[-0.200pt]{4.818pt}{0.400pt}}
\put(151.0,859.0){\rule[-0.200pt]{4.818pt}{0.400pt}}
\put(131,859){\makebox(0,0)[r]{$360$}}
\put(1419.0,859.0){\rule[-0.200pt]{4.818pt}{0.400pt}}
\put(151.0,131.0){\rule[-0.200pt]{0.400pt}{4.818pt}}
\put(151,90){\makebox(0,0){$2000$}}
\put(151.0,839.0){\rule[-0.200pt]{0.400pt}{4.818pt}}
\put(312.0,131.0){\rule[-0.200pt]{0.400pt}{4.818pt}}
\put(312,90){\makebox(0,0){$4000$}}
\put(312.0,839.0){\rule[-0.200pt]{0.400pt}{4.818pt}}
\put(473.0,131.0){\rule[-0.200pt]{0.400pt}{4.818pt}}
\put(473,90){\makebox(0,0){$6000$}}
\put(473.0,839.0){\rule[-0.200pt]{0.400pt}{4.818pt}}
\put(634.0,131.0){\rule[-0.200pt]{0.400pt}{4.818pt}}
\put(634,90){\makebox(0,0){$8000$}}
\put(634.0,839.0){\rule[-0.200pt]{0.400pt}{4.818pt}}
\put(795.0,131.0){\rule[-0.200pt]{0.400pt}{4.818pt}}
\put(795,90){\makebox(0,0){$10000$}}
\put(795.0,839.0){\rule[-0.200pt]{0.400pt}{4.818pt}}
\put(956.0,131.0){\rule[-0.200pt]{0.400pt}{4.818pt}}
\put(956,90){\makebox(0,0){$12000$}}
\put(956.0,839.0){\rule[-0.200pt]{0.400pt}{4.818pt}}
\put(1117.0,131.0){\rule[-0.200pt]{0.400pt}{4.818pt}}
\put(1117,90){\makebox(0,0){$14000$}}
\put(1117.0,839.0){\rule[-0.200pt]{0.400pt}{4.818pt}}
\put(1278.0,131.0){\rule[-0.200pt]{0.400pt}{4.818pt}}
\put(1278,90){\makebox(0,0){$16000$}}
\put(1278.0,839.0){\rule[-0.200pt]{0.400pt}{4.818pt}}
\put(1439.0,131.0){\rule[-0.200pt]{0.400pt}{4.818pt}}
\put(1439,90){\makebox(0,0){$18000$}}
\put(1439.0,839.0){\rule[-0.200pt]{0.400pt}{4.818pt}}
\put(151.0,131.0){\rule[-0.200pt]{0.400pt}{175.375pt}}
\put(151.0,131.0){\rule[-0.200pt]{310.279pt}{0.400pt}}
\put(1439.0,131.0){\rule[-0.200pt]{0.400pt}{175.375pt}}
\put(151.0,859.0){\rule[-0.200pt]{310.279pt}{0.400pt}}
\put(30,495){\makebox(0,0){\shortstack{Time\\ (nsec)}}}
\put(795,29){\makebox(0,0){Data set size}}
\put(1279,818){\makebox(0,0)[r]{'bst.dat' index 0}}
\put(155,193){\makebox(0,0){$\bullet$}}
\put(320,271){\makebox(0,0){$\bullet$}}
\put(485,563){\makebox(0,0){$\bullet$}}
\put(649,765){\makebox(0,0){$\bullet$}}
\put(814,812){\makebox(0,0){$\bullet$}}
\put(979,594){\makebox(0,0){$\bullet$}}
\put(1144,635){\makebox(0,0){$\bullet$}}
\put(1309,786){\makebox(0,0){$\bullet$}}
\put(1349,818){\makebox(0,0){$\bullet$}}
\put(1279,777){\makebox(0,0)[r]{54.6937 * log(x) - 183.644}}
\put(1299.0,777.0){\rule[-0.200pt]{24.090pt}{0.400pt}}
\put(155,201){\usebox{\plotpoint}}
\multiput(155.58,201.00)(0.492,0.798){21}{\rule{0.119pt}{0.733pt}}
\multiput(154.17,201.00)(12.000,17.478){2}{\rule{0.400pt}{0.367pt}}
\multiput(167.58,220.00)(0.492,0.826){19}{\rule{0.118pt}{0.755pt}}
\multiput(166.17,220.00)(11.000,16.434){2}{\rule{0.400pt}{0.377pt}}
\multiput(178.58,238.00)(0.492,0.712){21}{\rule{0.119pt}{0.667pt}}
\multiput(177.17,238.00)(12.000,15.616){2}{\rule{0.400pt}{0.333pt}}
\multiput(190.58,255.00)(0.492,0.732){19}{\rule{0.118pt}{0.682pt}}
\multiput(189.17,255.00)(11.000,14.585){2}{\rule{0.400pt}{0.341pt}}
\multiput(201.58,271.00)(0.492,0.669){21}{\rule{0.119pt}{0.633pt}}
\multiput(200.17,271.00)(12.000,14.685){2}{\rule{0.400pt}{0.317pt}}
\multiput(213.58,287.00)(0.492,0.582){21}{\rule{0.119pt}{0.567pt}}
\multiput(212.17,287.00)(12.000,12.824){2}{\rule{0.400pt}{0.283pt}}
\multiput(225.58,301.00)(0.492,0.637){19}{\rule{0.118pt}{0.609pt}}
\multiput(224.17,301.00)(11.000,12.736){2}{\rule{0.400pt}{0.305pt}}
\multiput(236.58,315.00)(0.492,0.539){21}{\rule{0.119pt}{0.533pt}}
\multiput(235.17,315.00)(12.000,11.893){2}{\rule{0.400pt}{0.267pt}}
\multiput(248.58,328.00)(0.492,0.539){21}{\rule{0.119pt}{0.533pt}}
\multiput(247.17,328.00)(12.000,11.893){2}{\rule{0.400pt}{0.267pt}}
\multiput(260.58,341.00)(0.492,0.543){19}{\rule{0.118pt}{0.536pt}}
\multiput(259.17,341.00)(11.000,10.887){2}{\rule{0.400pt}{0.268pt}}
\multiput(271.00,353.58)(0.543,0.492){19}{\rule{0.536pt}{0.118pt}}
\multiput(271.00,352.17)(10.887,11.000){2}{\rule{0.268pt}{0.400pt}}
\multiput(283.00,364.58)(0.543,0.492){19}{\rule{0.536pt}{0.118pt}}
\multiput(283.00,363.17)(10.887,11.000){2}{\rule{0.268pt}{0.400pt}}
\multiput(295.00,375.58)(0.496,0.492){19}{\rule{0.500pt}{0.118pt}}
\multiput(295.00,374.17)(9.962,11.000){2}{\rule{0.250pt}{0.400pt}}
\multiput(306.00,386.58)(0.600,0.491){17}{\rule{0.580pt}{0.118pt}}
\multiput(306.00,385.17)(10.796,10.000){2}{\rule{0.290pt}{0.400pt}}
\multiput(318.00,396.58)(0.600,0.491){17}{\rule{0.580pt}{0.118pt}}
\multiput(318.00,395.17)(10.796,10.000){2}{\rule{0.290pt}{0.400pt}}
\multiput(330.00,406.58)(0.547,0.491){17}{\rule{0.540pt}{0.118pt}}
\multiput(330.00,405.17)(9.879,10.000){2}{\rule{0.270pt}{0.400pt}}
\multiput(341.00,416.59)(0.669,0.489){15}{\rule{0.633pt}{0.118pt}}
\multiput(341.00,415.17)(10.685,9.000){2}{\rule{0.317pt}{0.400pt}}
\multiput(353.00,425.59)(0.669,0.489){15}{\rule{0.633pt}{0.118pt}}
\multiput(353.00,424.17)(10.685,9.000){2}{\rule{0.317pt}{0.400pt}}
\multiput(365.00,434.59)(0.611,0.489){15}{\rule{0.589pt}{0.118pt}}
\multiput(365.00,433.17)(9.778,9.000){2}{\rule{0.294pt}{0.400pt}}
\multiput(376.00,443.59)(0.758,0.488){13}{\rule{0.700pt}{0.117pt}}
\multiput(376.00,442.17)(10.547,8.000){2}{\rule{0.350pt}{0.400pt}}
\multiput(388.00,451.59)(0.758,0.488){13}{\rule{0.700pt}{0.117pt}}
\multiput(388.00,450.17)(10.547,8.000){2}{\rule{0.350pt}{0.400pt}}
\multiput(400.00,459.59)(0.692,0.488){13}{\rule{0.650pt}{0.117pt}}
\multiput(400.00,458.17)(9.651,8.000){2}{\rule{0.325pt}{0.400pt}}
\multiput(411.00,467.59)(0.758,0.488){13}{\rule{0.700pt}{0.117pt}}
\multiput(411.00,466.17)(10.547,8.000){2}{\rule{0.350pt}{0.400pt}}
\multiput(423.00,475.59)(0.758,0.488){13}{\rule{0.700pt}{0.117pt}}
\multiput(423.00,474.17)(10.547,8.000){2}{\rule{0.350pt}{0.400pt}}
\multiput(435.00,483.59)(0.798,0.485){11}{\rule{0.729pt}{0.117pt}}
\multiput(435.00,482.17)(9.488,7.000){2}{\rule{0.364pt}{0.400pt}}
\multiput(446.00,490.59)(0.874,0.485){11}{\rule{0.786pt}{0.117pt}}
\multiput(446.00,489.17)(10.369,7.000){2}{\rule{0.393pt}{0.400pt}}
\multiput(458.00,497.59)(0.874,0.485){11}{\rule{0.786pt}{0.117pt}}
\multiput(458.00,496.17)(10.369,7.000){2}{\rule{0.393pt}{0.400pt}}
\multiput(470.00,504.59)(0.798,0.485){11}{\rule{0.729pt}{0.117pt}}
\multiput(470.00,503.17)(9.488,7.000){2}{\rule{0.364pt}{0.400pt}}
\multiput(481.00,511.59)(0.874,0.485){11}{\rule{0.786pt}{0.117pt}}
\multiput(481.00,510.17)(10.369,7.000){2}{\rule{0.393pt}{0.400pt}}
\multiput(493.00,518.59)(1.033,0.482){9}{\rule{0.900pt}{0.116pt}}
\multiput(493.00,517.17)(10.132,6.000){2}{\rule{0.450pt}{0.400pt}}
\multiput(505.00,524.59)(0.798,0.485){11}{\rule{0.729pt}{0.117pt}}
\multiput(505.00,523.17)(9.488,7.000){2}{\rule{0.364pt}{0.400pt}}
\multiput(516.00,531.59)(1.033,0.482){9}{\rule{0.900pt}{0.116pt}}
\multiput(516.00,530.17)(10.132,6.000){2}{\rule{0.450pt}{0.400pt}}
\multiput(528.00,537.59)(1.033,0.482){9}{\rule{0.900pt}{0.116pt}}
\multiput(528.00,536.17)(10.132,6.000){2}{\rule{0.450pt}{0.400pt}}
\multiput(540.00,543.59)(0.943,0.482){9}{\rule{0.833pt}{0.116pt}}
\multiput(540.00,542.17)(9.270,6.000){2}{\rule{0.417pt}{0.400pt}}
\multiput(551.00,549.59)(1.033,0.482){9}{\rule{0.900pt}{0.116pt}}
\multiput(551.00,548.17)(10.132,6.000){2}{\rule{0.450pt}{0.400pt}}
\multiput(563.00,555.59)(1.033,0.482){9}{\rule{0.900pt}{0.116pt}}
\multiput(563.00,554.17)(10.132,6.000){2}{\rule{0.450pt}{0.400pt}}
\multiput(575.00,561.59)(1.155,0.477){7}{\rule{0.980pt}{0.115pt}}
\multiput(575.00,560.17)(8.966,5.000){2}{\rule{0.490pt}{0.400pt}}
\multiput(586.00,566.59)(1.033,0.482){9}{\rule{0.900pt}{0.116pt}}
\multiput(586.00,565.17)(10.132,6.000){2}{\rule{0.450pt}{0.400pt}}
\multiput(598.00,572.59)(1.155,0.477){7}{\rule{0.980pt}{0.115pt}}
\multiput(598.00,571.17)(8.966,5.000){2}{\rule{0.490pt}{0.400pt}}
\multiput(609.00,577.59)(1.267,0.477){7}{\rule{1.060pt}{0.115pt}}
\multiput(609.00,576.17)(9.800,5.000){2}{\rule{0.530pt}{0.400pt}}
\multiput(621.00,582.59)(1.033,0.482){9}{\rule{0.900pt}{0.116pt}}
\multiput(621.00,581.17)(10.132,6.000){2}{\rule{0.450pt}{0.400pt}}
\multiput(633.00,588.59)(1.155,0.477){7}{\rule{0.980pt}{0.115pt}}
\multiput(633.00,587.17)(8.966,5.000){2}{\rule{0.490pt}{0.400pt}}
\multiput(644.00,593.59)(1.267,0.477){7}{\rule{1.060pt}{0.115pt}}
\multiput(644.00,592.17)(9.800,5.000){2}{\rule{0.530pt}{0.400pt}}
\multiput(656.00,598.59)(1.267,0.477){7}{\rule{1.060pt}{0.115pt}}
\multiput(656.00,597.17)(9.800,5.000){2}{\rule{0.530pt}{0.400pt}}
\multiput(668.00,603.60)(1.505,0.468){5}{\rule{1.200pt}{0.113pt}}
\multiput(668.00,602.17)(8.509,4.000){2}{\rule{0.600pt}{0.400pt}}
\multiput(679.00,607.59)(1.267,0.477){7}{\rule{1.060pt}{0.115pt}}
\multiput(679.00,606.17)(9.800,5.000){2}{\rule{0.530pt}{0.400pt}}
\multiput(691.00,612.59)(1.267,0.477){7}{\rule{1.060pt}{0.115pt}}
\multiput(691.00,611.17)(9.800,5.000){2}{\rule{0.530pt}{0.400pt}}
\multiput(703.00,617.59)(1.155,0.477){7}{\rule{0.980pt}{0.115pt}}
\multiput(703.00,616.17)(8.966,5.000){2}{\rule{0.490pt}{0.400pt}}
\multiput(714.00,622.60)(1.651,0.468){5}{\rule{1.300pt}{0.113pt}}
\multiput(714.00,621.17)(9.302,4.000){2}{\rule{0.650pt}{0.400pt}}
\multiput(726.00,626.59)(1.267,0.477){7}{\rule{1.060pt}{0.115pt}}
\multiput(726.00,625.17)(9.800,5.000){2}{\rule{0.530pt}{0.400pt}}
\multiput(738.00,631.60)(1.505,0.468){5}{\rule{1.200pt}{0.113pt}}
\multiput(738.00,630.17)(8.509,4.000){2}{\rule{0.600pt}{0.400pt}}
\multiput(749.00,635.60)(1.651,0.468){5}{\rule{1.300pt}{0.113pt}}
\multiput(749.00,634.17)(9.302,4.000){2}{\rule{0.650pt}{0.400pt}}
\multiput(761.00,639.59)(1.267,0.477){7}{\rule{1.060pt}{0.115pt}}
\multiput(761.00,638.17)(9.800,5.000){2}{\rule{0.530pt}{0.400pt}}
\multiput(773.00,644.60)(1.505,0.468){5}{\rule{1.200pt}{0.113pt}}
\multiput(773.00,643.17)(8.509,4.000){2}{\rule{0.600pt}{0.400pt}}
\multiput(784.00,648.60)(1.651,0.468){5}{\rule{1.300pt}{0.113pt}}
\multiput(784.00,647.17)(9.302,4.000){2}{\rule{0.650pt}{0.400pt}}
\multiput(796.00,652.60)(1.651,0.468){5}{\rule{1.300pt}{0.113pt}}
\multiput(796.00,651.17)(9.302,4.000){2}{\rule{0.650pt}{0.400pt}}
\multiput(808.00,656.60)(1.505,0.468){5}{\rule{1.200pt}{0.113pt}}
\multiput(808.00,655.17)(8.509,4.000){2}{\rule{0.600pt}{0.400pt}}
\multiput(819.00,660.60)(1.651,0.468){5}{\rule{1.300pt}{0.113pt}}
\multiput(819.00,659.17)(9.302,4.000){2}{\rule{0.650pt}{0.400pt}}
\multiput(831.00,664.60)(1.651,0.468){5}{\rule{1.300pt}{0.113pt}}
\multiput(831.00,663.17)(9.302,4.000){2}{\rule{0.650pt}{0.400pt}}
\multiput(843.00,668.60)(1.505,0.468){5}{\rule{1.200pt}{0.113pt}}
\multiput(843.00,667.17)(8.509,4.000){2}{\rule{0.600pt}{0.400pt}}
\multiput(854.00,672.60)(1.651,0.468){5}{\rule{1.300pt}{0.113pt}}
\multiput(854.00,671.17)(9.302,4.000){2}{\rule{0.650pt}{0.400pt}}
\multiput(866.00,676.61)(2.472,0.447){3}{\rule{1.700pt}{0.108pt}}
\multiput(866.00,675.17)(8.472,3.000){2}{\rule{0.850pt}{0.400pt}}
\multiput(878.00,679.60)(1.505,0.468){5}{\rule{1.200pt}{0.113pt}}
\multiput(878.00,678.17)(8.509,4.000){2}{\rule{0.600pt}{0.400pt}}
\multiput(889.00,683.60)(1.651,0.468){5}{\rule{1.300pt}{0.113pt}}
\multiput(889.00,682.17)(9.302,4.000){2}{\rule{0.650pt}{0.400pt}}
\multiput(901.00,687.61)(2.472,0.447){3}{\rule{1.700pt}{0.108pt}}
\multiput(901.00,686.17)(8.472,3.000){2}{\rule{0.850pt}{0.400pt}}
\multiput(913.00,690.60)(1.505,0.468){5}{\rule{1.200pt}{0.113pt}}
\multiput(913.00,689.17)(8.509,4.000){2}{\rule{0.600pt}{0.400pt}}
\multiput(924.00,694.61)(2.472,0.447){3}{\rule{1.700pt}{0.108pt}}
\multiput(924.00,693.17)(8.472,3.000){2}{\rule{0.850pt}{0.400pt}}
\multiput(936.00,697.60)(1.651,0.468){5}{\rule{1.300pt}{0.113pt}}
\multiput(936.00,696.17)(9.302,4.000){2}{\rule{0.650pt}{0.400pt}}
\multiput(948.00,701.61)(2.248,0.447){3}{\rule{1.567pt}{0.108pt}}
\multiput(948.00,700.17)(7.748,3.000){2}{\rule{0.783pt}{0.400pt}}
\multiput(959.00,704.60)(1.651,0.468){5}{\rule{1.300pt}{0.113pt}}
\multiput(959.00,703.17)(9.302,4.000){2}{\rule{0.650pt}{0.400pt}}
\multiput(971.00,708.61)(2.472,0.447){3}{\rule{1.700pt}{0.108pt}}
\multiput(971.00,707.17)(8.472,3.000){2}{\rule{0.850pt}{0.400pt}}
\multiput(983.00,711.61)(2.248,0.447){3}{\rule{1.567pt}{0.108pt}}
\multiput(983.00,710.17)(7.748,3.000){2}{\rule{0.783pt}{0.400pt}}
\multiput(994.00,714.60)(1.651,0.468){5}{\rule{1.300pt}{0.113pt}}
\multiput(994.00,713.17)(9.302,4.000){2}{\rule{0.650pt}{0.400pt}}
\multiput(1006.00,718.61)(2.248,0.447){3}{\rule{1.567pt}{0.108pt}}
\multiput(1006.00,717.17)(7.748,3.000){2}{\rule{0.783pt}{0.400pt}}
\multiput(1017.00,721.61)(2.472,0.447){3}{\rule{1.700pt}{0.108pt}}
\multiput(1017.00,720.17)(8.472,3.000){2}{\rule{0.850pt}{0.400pt}}
\multiput(1029.00,724.61)(2.472,0.447){3}{\rule{1.700pt}{0.108pt}}
\multiput(1029.00,723.17)(8.472,3.000){2}{\rule{0.850pt}{0.400pt}}
\multiput(1041.00,727.61)(2.248,0.447){3}{\rule{1.567pt}{0.108pt}}
\multiput(1041.00,726.17)(7.748,3.000){2}{\rule{0.783pt}{0.400pt}}
\multiput(1052.00,730.60)(1.651,0.468){5}{\rule{1.300pt}{0.113pt}}
\multiput(1052.00,729.17)(9.302,4.000){2}{\rule{0.650pt}{0.400pt}}
\multiput(1064.00,734.61)(2.472,0.447){3}{\rule{1.700pt}{0.108pt}}
\multiput(1064.00,733.17)(8.472,3.000){2}{\rule{0.850pt}{0.400pt}}
\multiput(1076.00,737.61)(2.248,0.447){3}{\rule{1.567pt}{0.108pt}}
\multiput(1076.00,736.17)(7.748,3.000){2}{\rule{0.783pt}{0.400pt}}
\multiput(1087.00,740.61)(2.472,0.447){3}{\rule{1.700pt}{0.108pt}}
\multiput(1087.00,739.17)(8.472,3.000){2}{\rule{0.850pt}{0.400pt}}
\multiput(1099.00,743.61)(2.472,0.447){3}{\rule{1.700pt}{0.108pt}}
\multiput(1099.00,742.17)(8.472,3.000){2}{\rule{0.850pt}{0.400pt}}
\multiput(1111.00,746.61)(2.248,0.447){3}{\rule{1.567pt}{0.108pt}}
\multiput(1111.00,745.17)(7.748,3.000){2}{\rule{0.783pt}{0.400pt}}
\multiput(1122.00,749.61)(2.472,0.447){3}{\rule{1.700pt}{0.108pt}}
\multiput(1122.00,748.17)(8.472,3.000){2}{\rule{0.850pt}{0.400pt}}
\put(1134,752.17){\rule{2.500pt}{0.400pt}}
\multiput(1134.00,751.17)(6.811,2.000){2}{\rule{1.250pt}{0.400pt}}
\multiput(1146.00,754.61)(2.248,0.447){3}{\rule{1.567pt}{0.108pt}}
\multiput(1146.00,753.17)(7.748,3.000){2}{\rule{0.783pt}{0.400pt}}
\multiput(1157.00,757.61)(2.472,0.447){3}{\rule{1.700pt}{0.108pt}}
\multiput(1157.00,756.17)(8.472,3.000){2}{\rule{0.850pt}{0.400pt}}
\multiput(1169.00,760.61)(2.472,0.447){3}{\rule{1.700pt}{0.108pt}}
\multiput(1169.00,759.17)(8.472,3.000){2}{\rule{0.850pt}{0.400pt}}
\multiput(1181.00,763.61)(2.248,0.447){3}{\rule{1.567pt}{0.108pt}}
\multiput(1181.00,762.17)(7.748,3.000){2}{\rule{0.783pt}{0.400pt}}
\put(1192,766.17){\rule{2.500pt}{0.400pt}}
\multiput(1192.00,765.17)(6.811,2.000){2}{\rule{1.250pt}{0.400pt}}
\multiput(1204.00,768.61)(2.472,0.447){3}{\rule{1.700pt}{0.108pt}}
\multiput(1204.00,767.17)(8.472,3.000){2}{\rule{0.850pt}{0.400pt}}
\multiput(1216.00,771.61)(2.248,0.447){3}{\rule{1.567pt}{0.108pt}}
\multiput(1216.00,770.17)(7.748,3.000){2}{\rule{0.783pt}{0.400pt}}
\put(1227,774.17){\rule{2.500pt}{0.400pt}}
\multiput(1227.00,773.17)(6.811,2.000){2}{\rule{1.250pt}{0.400pt}}
\multiput(1239.00,776.61)(2.472,0.447){3}{\rule{1.700pt}{0.108pt}}
\multiput(1239.00,775.17)(8.472,3.000){2}{\rule{0.850pt}{0.400pt}}
\multiput(1251.00,779.61)(2.248,0.447){3}{\rule{1.567pt}{0.108pt}}
\multiput(1251.00,778.17)(7.748,3.000){2}{\rule{0.783pt}{0.400pt}}
\put(1262,782.17){\rule{2.500pt}{0.400pt}}
\multiput(1262.00,781.17)(6.811,2.000){2}{\rule{1.250pt}{0.400pt}}
\multiput(1274.00,784.61)(2.472,0.447){3}{\rule{1.700pt}{0.108pt}}
\multiput(1274.00,783.17)(8.472,3.000){2}{\rule{0.850pt}{0.400pt}}
\put(1286,787.17){\rule{2.300pt}{0.400pt}}
\multiput(1286.00,786.17)(6.226,2.000){2}{\rule{1.150pt}{0.400pt}}
\multiput(1297.00,789.61)(2.472,0.447){3}{\rule{1.700pt}{0.108pt}}
\multiput(1297.00,788.17)(8.472,3.000){2}{\rule{0.850pt}{0.400pt}}
\put(151.0,131.0){\rule[-0.200pt]{0.400pt}{175.375pt}}
\put(151.0,131.0){\rule[-0.200pt]{310.279pt}{0.400pt}}
\put(1439.0,131.0){\rule[-0.200pt]{0.400pt}{175.375pt}}
\put(151.0,859.0){\rule[-0.200pt]{310.279pt}{0.400pt}}
\end{picture}

  \caption {BST 의 검색 동작 시간}
  \label{fig:bst}
\end {figure}

\section {벤치마크에 사용된 코드}
테스트 환경:
OS: Gentoo Linux 
C compiler: GCC 4.9.2 with -std=c99
Make: GNU Make 4.1
에서 테스트 했으며, 코드는 C99 표준을 이용하여 작성되었습니다.

메인 소스 코드
\lstinputlisting [style=customc]{project/integerapp.c}

이 코드에서는, ADT의 함수들을 static 으로 선언하고, 해당 함수들의 Function pointer 를 구조체에 담아, 객체를 통해 메서드를 호출하도록 구성하였습니다. BST 의 경우, 교과서의 구현과 같고, Linked list 의 경우, 재귀 호출을 사용하는 구현으로 바꾸어 구현하였습니다.

BST ADT
\lstinputlisting [style=customc]{project/bst.h}

LinkedList ADT
\lstinputlisting [style=customc]{project/linkedlist.h}

BST 구현
\lstinputlisting [style=customc]{project/bst.c}

LinkedList 구현
\lstinputlisting [style=customc]{project/linkedlist.c}

\end {document}
